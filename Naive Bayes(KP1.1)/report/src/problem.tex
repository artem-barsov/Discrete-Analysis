\CWHeader{Курсовой проект}

\CWProblem{
Реализуйте систему, которая на основе базы вопросов и тегов к ним, будет предлагать варианты тегов, которые подходят к новым вопросам.

Формат запуска программы в режиме обучения: \\
\texttt{
./prog learn -{}-input <input file> -{}-output <stats file>
} \\ \\
\hspace{20pt}\begin{tabular}{|l|l|}
    \hline
    Ключ & Значение \\
    \hline\hline
    -{}-input & входной файл с вопросами \\
    \hline
    -{}-output & выходной файл с рассчитанной статистикой \\
    \hline
\end{tabular} \\

Формат запуска программы в режиме классификации: \\
\texttt{
./prog classify -{}-stats <stat file> -{}-input <in file> -{}-output <out file>
} \\
\hspace{20pt}\begin{tabular}{|l|l|}
    \hline
    Ключ & Значение \\
    \hline\hline
    -{}-stats & файл со статистикой полученной на предыдущем этапе \\
    \hline
    -{}-input & входной файл с вопросами \\
    \hline
    -{}-output & выходной файл с тегами к вопросам \\
    \hline
\end{tabular} \\

\noindent{\bfseries Формат входных файлов при обучении: } \\
\texttt{
<Количество строк в вопросе [n]> \\
<Тег 1>,<Тег 2>,...,<Тег m> \\
<Заголовок вопроса> \\
<Текст вопроса [n строк]>
} \\ \\
{\bfseries Формат входных файлов при запросах: } \\
\texttt{
<Количество строк в вопросе [n]> \\
<Заголовок вопроса> \\
<Текст вопроса [n строк]>
} \\ \\
{\bfseries Формат выходного файла: } \\
Для каждого запроса в отдельной строке выводится предполагаемый набор тегов, через запятую. \\ \\

}
\pagebreak
